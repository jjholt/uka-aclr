\Gls{oa} of the knee is the primary indicator for arthroplasty, being the sole indicator in over 96\% of cases \cite{achakri_national_2022}.
While half of patients in England and Wales have isolated unicompartmental knee disease \cite{stoddart_compartmental_2021}, only about one-in-eight undergo \gls{uka} \cite{achakri_national_2022}.

In spite of \gls{uka} having more than double the revision rate compared to \gls{tka} (10.52\% and 4.24\%, respectively) \cite{achakri_national_2022}, they are associated with better health and functional outcomes \cite{liddle_adverse_2014,blevins_postoperative_2020}.
% In part, this is due to the lower barrier to consider a revision surgery for a \gls{uka} with poor outcomes \cite{goodfellow_critique_2010}.

Given \gls{uka}'s relative underutilisation and higher revision rates, there is great value in research that could extend its indications while reducing the risk of revision.

\subsection{Why focus on the ACL?}
The target population is of advanced age and has osteoarthritic knees, which is associated with ACL insufficiency \cite{scott_patterns_2020}. 
Historically, absence of the \gls{acl} was a contraindication for \gls{uka}, however some now consider this contraindication to be obsolete \cite{van_der_list_role_2016}.

A little over half of knees have macroscopically intact ACLs preceding arthroplasty \cite{roussi_anterior_2021}, though that does not imply they are healthy.
An MRI study found microscopic abnormalities in 33\% of macroscopically intact ligaments, however fewer than 15\% were found to be unsuitable during intraoperative inspection \cite{sharpe_magnetic_2001}.
Histological data is even more extreme --- where one study found microscopic damage (i.e., loss of compaction and/or presence of fibroblast nuclei) in 70\% of macroscopically intact ligaments \cite{trompeter_predicting_2009}.

% The \gls{acl} is the primary restraint for anterior drawer \cite{butler_ligamentous_1980}. Naturally, its absence has been shown to increase tibial anterior laxity \cite{yagi_biomechanical_2002}.
While most data consider the effect of ACL deficiency on mobile bearing UKA, previous work into fixed-bearing showed that \gls{uka} alone could not restore function of a lost \gls{acl} \cite{suggs_knee_2006}.
In fact, it is speculated that part of the extensor moment efficiency gained by \gls{uka} over \gls{tka} is caused by the presence of an \gls{acl} \cite{garner_extensor_2021}.

In summary, patients are undergoing \gls{uka} with varying degrees of degeneration of their \gls{acl}.
By better understanding fixed-bearing \gls{uka} biomechanics in the presence of deficient (missing), insufficient (present, but damaged) and reconstructed \gls{acl}s, it may be possible to increase utilisation without impacting revision rates. 

% \todo[inline]{Missing why we care about micromotion and looking at different slopes?
%
%     Slope: Increase PTS => anterior tibial translation
%
%     Micromotion: Reduces osseointegration
% }

% \section{Experimental rig (Robot)}
% \paragraph{Why?}
% \paragraph{Idea behind it}

% \subsection{Alternative marginal interventions}
% \paragraph{Lateral Extra-articular Tenodesis (LET)}
% Chronic insufficiency of the ACL means non-physiological loading and subsequent weakening of the lateral structures.
% LET provides rotational stability by supporting those lateral structures and, at most, some marginal sagittal off-loading of the ACL \cite{marom_lateral_2020}.
% Despite only possible marginal benefit, it is worth exploring its benefit because the surgical technique is so much simpler than intra-articular augmentation \cite{abusleme_lateral_2021, alkhafaji_modified_2022}.
%
% A similar set of modest benefits in AP laxity can be seen from a reconstruction of the MCL \cite{butler_ligamentous_1980}.
% In rotation, the forces experienced by the POL are about half of the ACL \cite{schafer_distribution_2016}.
% While this provides some reason for a medial extra-articular intervention and the addition of rotational tests, these provide less meaningful insights than the alternatives.
%

% \subsection{Controls}
% \paragraph{Decline in tissue quality}
% \todo[inline]{
%     Current protocol doesn't really have any space for randomisation.
% }
