This experiment aims to explore the biomechanics of \gls{uka} in the presence of an insufficient, deficient and reconstructed \gls{acl}.
Additionally for \gls{acl} deficient and reconstructed knees, it will investigate how \gls{pts} affects the biomechanics, as well as the micromotion undergone by the implant.

10 cadaveric knee specimens will be tested in a six degree-of-freedom robotic system with the objective of establishing a stability envelope.
The robot will apply body weight compression across the joint while flexing and extending the knee, with and without \gls{ap} loading ($\pm \SI{90}{N}$) and \gls{ie} torque ($\pm \SI{5}{Nm}$).
Kinematics will be analysed for the following states:
\begin{enumerate}
    \item \textbf{Native knee:} Grade \gls{ap} laxity.
    \item \textbf{\gls{uka}} with a normal \gls{pts} (about 4\degree{}).
    % \item Place fixture to change posterior slope to 15\degree{}, 10\degree{}, 5\degree{}, 0\degree{}. Re-sit micromotion rig. Measure kinematics and implant micromotion.
    \item \textbf{ACL insufficiency:} Pie-crust (or partially transect) \gls{acl} so \gls{ap} laxity is \SI{10}{mm}.
\item \textbf{ACL augmentation:} Implant synthetic ligament with the insufficient \gls{acl} still in place.
    \item \textbf{ACL replacement:} Release the ACL with the synthetic ligament still in place.
    \item \textbf{ACL deficient:} Remove synthetic ligament.
    \item Increase tibial slope to 12\degree{} and repeat steps 5 and 6.
\end{enumerate}

As a stretch goal, mm-scale fixtures will be attached to the implants to enable the measurement of tibial tray micromotion using LVDTs across all testing states.
