Both kinematic and kinetic data are acquired using a SimVitro-controlled 6-DOF robotic manipulator (KUKA, Augsburg, Germany).
Manipulating the cadaver using displacement-control requires certain assumptions about how well the structures can survive the displacements before rupturing. 
Therefore, knee flexion is performed under force-control based on the somewhat reliable KT--1000 arthrometer, which applies \SI{89}{\newton} tibial anterior force \cite{wiertsema_reliability_2008}.

After modifying the knee, the robot replicates the same manipulation but in displacement control.
The difference in forces is attributed to the modified structure by the principle of superposition \cite{hoher_situ_1999}.
Similar approach has been used to study knee biomechanics in \gls{uka} in relation to the \gls{acl} \cite{suggs_knee_2006}, as well as the function of the \gls{pcl} \cite{hoher_situ_1999}, \gls{all} \cite{parsons_biomechanical_2015} and more.
